\documentclass[a4paper,11pt]{article}
\usepackage{amssymb}
\usepackage{svg}
\usepackage{relsize}
\usepackage{gensymb}
\usepackage{natbib}
\usepackage{float}
\usepackage[polish]{babel}
\usepackage[utf8]{inputenc}
\usepackage[T1]{fontenc}
\usepackage{graphicx}
\usepackage{anysize}
\usepackage{enumerate}
\usepackage{times}
\usepackage{geometry}
\usepackage{amsthm}
\usepackage{pgfplots}
\usepackage{makecell}
\usepackage{caption}


\usepackage[intlimits]{amsmath}
\marginsize{3cm}{3cm}{1.5cm}{1.5cm}
\sloppy

\begin{document}
\begin{table}[ht]
\centering
\hspace*{-1cm}
\begin{tabular}{lllllll}
\cline{1-6}
\multicolumn{1}{|c|}{\begin{tabular}[c]{@{}c@{}}EAIiIB\\ Informatyka\end{tabular}}              & \multicolumn{2}{l|}{\begin{tabular}[c]{@{}l@{}}Ewa Stachów\\ Weronika Olcha\end{tabular}}                                                                                                & \multicolumn{1}{c|}{\begin{tabular}[c]{@{}c@{}}Rok\\ II\end{tabular}}          & \multicolumn{1}{c|}{\begin{tabular}[c]{@{}c@{}}Grupa\\ 3\end{tabular}}            & \multicolumn{1}{c|}{\begin{tabular}[c]{@{}c@{}}Zespół\\ 6\end{tabular}}      &  \\ \cline{1-6}
\multicolumn{1}{|c|}{\begin{tabular}[c]{@{}c@{}}Pracownia\\ FIZYCZNA\\ WFiIS AGH\end{tabular}} & \multicolumn{4}{l|}{\begin{tabular}[c]{@{}l@{}}Temat:\\ \textbf{\textit{Fale podłużne w ciałach stałych}} \end{tabular}}                                                                                                                                                                                                                                            & \multicolumn{1}{c|}{\begin{tabular}[c]{@{}c@{}}Nr ćwiczenia:\\ 29\end{tabular}} &  \\ \cline{1-6}
\multicolumn{1}{|l|}{\begin{tabular}[c]{@{}c@{}}Data wykonania:\\ 27.11.2016\end{tabular}}      & \multicolumn{1}{c|}{\begin{tabular}[c]{@{}c@{}}Data oddania:\\ 05.12.2016\end{tabular}} & \multicolumn{1}{l|}{\begin{tabular}[c]{@{}l@{}}Zwrot do poprawki:\\ \phantom{data poprawki}\end{tabular}} & \multicolumn{1}{l|}{\begin{tabular}[c]{@{}l@{}}Data oddania:\\  \phantom{data oddania}\end{tabular}} & \multicolumn{1}{l|}{\begin{tabular}[c]{@{}l@{}}Data zaliczenia:\\  \phantom{data zaliczenia}\end{tabular}} & \multicolumn{1}{l|}{\begin{tabular}[c]{@{}l@{}}OCENA:\\ \phantom{ocena}\end{tabular}}       &  \\ \cline{1-6}
                                                                                               &                                                                                         &                                                                                     &                                                                                &                                                                                   &                                                                               & 
\end{tabular}
\end{table}

\begin{center}
\begin{LARGE}
\textbf{Ćwiczenie nr 29: Fale podłużne w ciałach stałych}
\end{LARGE}
\end{center}

\section{Cel ćwiczenia}
Wyznaczenie modułu Younga dla różnych materiałów na podstawie pomiaru prędkości rozchodzenia się fali.

\section{Część teoretyczna}
 
Fala podłużna  to fala, w której drgania odbywają się w kierunku zgodnym z kierunkiem jej rozchodzenia się. 
Opisuje ją równanie $$ y= Acos(\omega t \pm kx) $$
Prawa Hooke'a: Odkształcenie jest wprost proporcjonalne do wywołującej je siły.
Opisuje to wzór: $$ \Delta l = \frac{Fl}{ES} $$
$\Delta l$ - zmiana długości pręta, F - siła odkształcająca, l - długość, S - pole przekroju.
Współczynnik E to właśnie stała nazwana modułem Younga.
Wyprowadzenie wzoru na moduł Younga, który będzie przydatny do późniejszych obliczeń.
Wychodząc od ogólnego wzoru na prawo Hooke'a: 
$$ \sigma = \varepsilon E $$  $\sigma $ - naprężenie, $\varepsilon $ - odkształcenie względne
$$ \varepsilon = \frac{\delta \Psi}{\delta x} $$
Otrzymujemy wzór na  prędkość rozchodzenia się fali w pręcie: 
$$ v = \sqrt{\frac{E}{\rho}} $$
czyli $$ E = v^2\rho $$
W pręcie powstaje fala stojąca, odległość między węzłami fali stojące wynosi $ l = \frac{1}{2} \lambda $, z tego obliczamy prędkość rozchodzenia się fali $ v = 2lf $ , f - częstotliwość fali.
Podstawiając to wcześniejszego wzoru ostatecznie otrzymujemy:
$$\mathlarger{E = 4\rho f^2 l^2}$$
 
 
\section{Przebieg ćwiczenia}

Układ pomiarowy składa się ze stojaka z prętami i rurami zawieszonymi na niciach, wagi, młotka, śruby mikrometrowej i komputera z mikrofonem z zainstalowanym oprogramowaniem Zelscope.

Przebieg doświadczenia:
\begin{enumerate}

    \item Zważenie pręta lub rury, dokonanie pomiarów długości i wymiarów podstawy, ustalenie rodzaju materiału z jakiego wykonany jest badany obiekt.
    \item Wyznaczenie za pomocą młotka i programu składowych harmonicznych dla badanego obiektu
    \item Powtórzenie procedury dla kolejnych obiektów.
     
            
\end{enumerate}

%pierwsza tabela z wynikami
\section{Wyniki pomiarów}

\begin{table}[!htbp]
\centering
\resizebox{\columnwidth}{!}{
\begin{tabular}{ll|l|l|}
\hline
\multicolumn{4}{|c|}{\textbf{PRĘT 1 (MIEDŹ)}}                                                                                                                                  \\ \hline
\multicolumn{1}{|l|}{\textbf{Długość l {[}$m${]}}}       & 1.8                               & \textbf{Masa m {[}$kg${]}}                         & 0.067                              \\ \hline
\multicolumn{1}{|l|}{\textbf{Objętość {[}$m^3${]}}}      & 0.0000075                           & \textbf{Gęstość ro {[}$\frac{kg}{m^3}${]}} & 8886,15 \\ \hline
\multicolumn{1}{|l|}{\textbf{NR HARMONICZNEJ}}         & \textbf{CZĘSTOTLIWOŚĆ f {[}$Hz${]}} & \textbf{DŁUGOŚĆ FALI $\lambda$ {[}$m${]}}             & \textbf{PRĘDKOŚĆ FALI v {[}$\frac{m}{s}${]}} \\ \hline
\multicolumn{1}{|l|}{1} & 1031.25  & 3.6  & 3712,5 \\ \hline
\multicolumn{1}{|l|}{2} & 2062,5 & 1.8 & 3712,5  \\ \hline
\multicolumn{1}{|l|}{3} & 3093,75 & 1.2 & 3712,5 \\ \hline
\multicolumn{1}{|l|}{4} & 4125 & 0.9 & 3712,5\\ \hline
\multicolumn{1}{|l|}{5} & 5156,25 & 0.72 & 3712,5 \\ \hline
\multicolumn{1}{|l|}{6} & 6187,5  & 0.6 & 3712,5\\ \hline
&  & \textbf{ŚREDNIA PRĘDKOŚĆ v {[}$\frac{m}{s}${]}} & 3712,5 \\ \cline{3-4} & & \textbf{MODUŁ YOUNGA {[}$GPa${]}} & 122.48 \\ \cline{3-4} 
\end{tabular}
}
\end{table}

\begin{table}[!htbp]
\resizebox{\textwidth}{!}{%
\begin{tabular}{ll|l|l|}
\hline
\multicolumn{4}{|c|}{\textbf{PRĘT 2 (STAL)}}                                                                                                                                    \\ \hline
\multicolumn{1}{|l|}{\textbf{Długość l {[}$m${]}}}       & 1.8                               & \textbf{Masa m {[}$kg${]}}                         & 0.033                              \\ \hline
\multicolumn{1}{|l|}{\textbf{Objętość {[}$m^3${]}}}      & 0.000004                           & \textbf{Gęstość ro {[}$\frac{kg}{m^3}${]}} & 8243,41 \\ \hline
\multicolumn{1}{|l|}{\textbf{NR HARMONICZNEJ}}         & \textbf{CZĘSTOTLIWOŚĆ f {[}$Hz${]}} & \textbf{DŁUGOŚĆ FALI $\lambda$ {[}$m${]}}             & \textbf{PRĘDKOŚĆ FALI v {[}$\frac{m}{s}${]}} \\ \hline
\multicolumn{1}{|l|}{1} & 1429,69  & 3.6  & 5146,884 \\ \hline
\multicolumn{1}{|l|}{2} & 2859,38 & 1.8 & 5146,884  \\ \hline
\multicolumn{1}{|l|}{3} & 4289,06 & 1.2 & 5146,872 \\ \hline
\multicolumn{1}{|l|}{4} & 5742,19 & 0.9 & 5167,971\\ \hline
\multicolumn{1}{|l|}{5} & 7171,88 & 0.72 & 5163,75 \\ \hline
\multicolumn{1}{|l|}{6} & 8601,56  & 0.6 & 5160,94\\ \hline
&  & \textbf{ŚREDNIA PRĘDKOŚĆ v {[}$\frac{m}{s}${]}} & 5155,55 \\ \cline{3-4} & & \textbf{MODUŁ YOUNGA {[}$GPa${]}} & 219,11 \\ \cline{3-4} 
\end{tabular}%
}
\end{table}

\begin{table}[!htbp]
\resizebox{\textwidth}{!}{%
\begin{tabular}{ll|l|l|}
\hline
\multicolumn{4}{|c|}{\textbf{PRĘT 3 (MOSIĄDZ)}}                                                                                                                                    \\ \hline
\multicolumn{1}{|l|}{\textbf{Długość l {[}$m${]}}}       & 1 & \textbf{Masa m {[}$kg${]}}                         & 0.175                              \\ \hline
\multicolumn{1}{|l|}{\textbf{Objętość {[}$m^3${]}}}      & 0.00002                           & \textbf{Gęstość ro {[}$\frac{kg}{m^3}${]}} & 8741,26 \\ \hline
\multicolumn{1}{|l|}{\textbf{NR HARMONICZNEJ}}         & \textbf{CZĘSTOTLIWOŚĆ f {[}$Hz${]}} & \textbf{DŁUGOŚĆ FALI $\lambda$ {[}$m${]}}             & \textbf{PRĘDKOŚĆ FALI v {[}$\frac{m}{s}${]}} \\ \hline
\multicolumn{1}{|l|}{1} & 1710,94 & 2 & 3421,88 \\ \hline
\multicolumn{1}{|l|}{2} & 3445,31 & 1 & 3445,31  \\ \hline
\multicolumn{1}{|l|}{3} & 5156,25 & 0,67 & 3437,5 \\ \hline
\multicolumn{1}{|l|}{4} & 6890,63 & 0,5 & 3445,315\\ \hline
\multicolumn{1}{|l|}{5} & 8601,56 & 0,4 & 3440,624 \\ \hline
&  & \textbf{ŚREDNIA PRĘDKOŚĆ v {[}$\frac{m}{s}${]}} & 3438,1258 \\ \cline{3-4} & & \textbf{MODUŁ YOUNGA {[}$GPa${]}} & 103,33 \\ \cline{3-4}
\end{tabular}%
}
\end{table}

\begin{table}[!htbp]
\resizebox{\textwidth}{!}{%
\begin{tabular}{ll|l|l|}
\hline
\multicolumn{4}{|c|}{\textbf{PRĘT 4 (ALUMINIUM)}}                                                                                                                                 \\ \hline
\multicolumn{1}{|l|}{\textbf{Długość l {[}$m${]}}}       & 1 & \textbf{Masa m {[}$kg${]}}                         & 0.024                              \\ \hline
\multicolumn{1}{|l|}{\textbf{Objętość {[}$m^3${]}}}      & 0.0000086                           & \textbf{Gęstość ro {[}$\frac{kg}{m^3}${]}} & 2790,66 \\ \hline
\multicolumn{1}{|l|}{\textbf{NR HARMONICZNEJ}}         & \textbf{CZĘSTOTLIWOŚĆ f {[}$Hz${]}} & \textbf{DŁUGOŚĆ FALI $\lambda$ {[}$m${]}}             & \textbf{PRĘDKOŚĆ FALI v {[}$\frac{m}{s}${]}} \\ \hline
\multicolumn{1}{|l|}{1} & 2460,94 &	2 & 4921,88 \\ \hline
\multicolumn{1}{|l|}{2} & 4945,31 & 1 & 4945,31  \\ \hline
\multicolumn{1}{|l|}{3} & 7406,25 & 0,67 & 4937,5 \\ \hline
\multicolumn{1}{|l|}{4} & 9867,19 & 0,5 & 4933,595\\ \hline
&  & \textbf{ŚREDNIA PRĘDKOŚĆ v {[}$\frac{m}{s}${]}} & 4934,57 \\ \cline{3-4} & & \textbf{MODUŁ YOUNGA {[}$GPa${]}} & 67,85 \\ \cline{3-4}
\end{tabular}%
}
\end{table}


\newpage
\section{Opracowanie wyników}
	Dla obliczeń błędów pomiaru przyjęto następujące niepewności:\\
	Dla długości pręta: $u(l)=1 [mm]$\\
	Dla długości mierzonych suwmiarką: $u(s)=0,1[mm]$\\
	Dla masy próbki:$u(m)=1 [g]$\\
	Dla częstotliwości:$u(f)=20 [Hz]$\\
	$$$$
	\textbf{Niepewność gęstości:}
	$$ u(\rho)=\sqrt{\bigg(\frac{\partial \rho}{\partial m}u(m)\bigg)^2+\bigg(\frac{\partial \rho}{\partial l}u(l)\bigg)^2+\bigg(\frac{\partial \rho}{\partial r}u(r)\bigg)^2+\bigg(\frac{\partial \rho}{\partial R}u(R)\bigg)^2} = $$ $$ = \sqrt{\bigg(\frac{1}{l\Pi (R^2-r^2)}u(m)\bigg)^2+\bigg(\frac{-m}{l^2 \Pi(R^2-r^2)}u(l)\bigg)^2+\bigg(\frac{-2mr}{l\Pi (R^2-r^2)^2}u(r)\bigg)^2+\bigg(\frac{-2mR}{l\Pi (R^2-r^2)^2}u(R)\bigg)^2}$$
	$$$$
\textbf{Niepewność długości fali:}
	$$ u(\lambda)=\sqrt{\bigg(\frac{2}{n}u(l)\bigg)^2}$$
	\textbf{Niepewność prędkości fali:}
	$$ u(v)=\sqrt{\bigg(\frac{\partial v}{\partial f}u(f)\bigg)^2+\bigg(\frac{\partial v}{\partial \lambda}u(\lambda)\bigg)^2}=\sqrt{\bigg(\lambda u(f)\bigg)^2+\bigg(f u(\lambda)\bigg)^2}$$
	$$$$
	\textbf{Niepewność modułu Younga:}
	$$ u(E)=\sqrt{\bigg(\frac{\partial E}{\partial \rho}u(\rho)\bigg)^2+\bigg(\frac{\partial E}{\partial v}u(v)\bigg)^2} =
	\sqrt{\bigg(v^2 u(\rho)\bigg)^2+\bigg(2 \rho v u(v)\bigg)^2}$$
	
\begin{table}[!htbp]
\resizebox{\textwidth}{!}{%
\begin{tabular}{|l|l|l|l|}
\hline
\multicolumn{1}{|c|}{\textbf{Nr pręta (materiał)}} & \multicolumn{1}{c|}{\textbf{Niepewność gęstości u(ro) {[}$\frac{kg}{m^3}$]}} & \multicolumn{1}{c|}{\textbf{Niepewność prędkości fali u(v) {[}$\frac{m}{s}${]}}} & \multicolumn{1}{c|}{\textbf{Niepewność moduług Younga u(E) {[}$GPa${]}}} \\ \hline
1 (miedź)                                      & 1421,81
 & 72,01
 & 20,16                                                                 \\ \hline
2 (stal)                                           & 92,16 
& 72,01
 & 6,59 \\ \hline
3 (mosiądz)                                        & 129,94                                                                            & 40,04                                                                & 2,86
 \\ \hline
4 (aluminium)                                           & 446,51                                                                           & 40,0 & 
10.93 \\ \hline
\end{tabular}%
}
\end{table}
\section{Porównanie wyników z wartościami tabelarycznymi}
$E_{t}$ -- wartość tabelaryczna modułu Younga, 
$E_{w}$ -- wartość wyliczona modułu Younga \newline
\textbf{MIEDŹ}
$$|E_{t}-E_{w}|=|117-122,48|~[GPa] < 2 \cdot 20,16~[GPa]$$
\textbf{STAL}
$$|E_{t}-E_{w}|=|200-219,11|~[GPa] >  2 \cdot 6,59~[GPa]$$
\textbf{MOSIĄDZ}
$$|E_{t}-E_{w}|=|103-103,33|~[GPa] <  2 \cdot 2,86~[GPa]$$
\textbf{ALUMINIUM}
$$|E_{t}-E_{w}|=|69-67,85|~[GPa]<  2 \cdot 10,93 ~[GPa]$$

\section{Wnioski}

Na podstawie wymiarów pręta oraz pomiaru częstotliwości przy pomocy programu Zelscope wyznaczyłyśmy gęstość materiału oraz prędkość rozchodzenia się w nim fali. Dzięki temu obliczyłyśmy wartość modułu Younga oraz niepewność standardową wartości modułu Younga dla każdego z materiałów. Prawie wszystkie wyznaczone wartości modułu Younga zgadzają się z wartościami tabelarycznymi, wyjątek stanowi stal (możliwy błąd w pomiarach materiału, błąd rachunkowy).


\end{document}
