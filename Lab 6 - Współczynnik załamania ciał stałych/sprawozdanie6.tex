\documentclass[a4paper,11pt]{article}
\usepackage{amssymb}
\usepackage[polish]{babel}
\usepackage[utf8]{inputenc}
\usepackage[T1]{fontenc}
\usepackage{graphicx}
\usepackage{anysize}
\usepackage{enumerate}
\usepackage{times}
\usepackage{geometry}
\usepackage{amsthm}
\usepackage{pgfplots}

\usepackage[intlimits]{amsmath}
\marginsize{3cm}{3cm}{1.5cm}{1.5cm}
\sloppy

\begin{document}
\begin{table}[ht]
\centering
\hspace*{-1cm}
\begin{tabular}{lllllll}
\cline{1-6}
\multicolumn{1}{|c|}{\begin{tabular}[c]{@{}c@{}}EAIiIB\\ Informatyka\end{tabular}}              & \multicolumn{2}{l|}{\begin{tabular}[c]{@{}l@{}}Ewa Stachów\\ Weronika Olcha\end{tabular}}                                                                                                & \multicolumn{1}{c|}{\begin{tabular}[c]{@{}c@{}}Rok\\ II\end{tabular}}          & \multicolumn{1}{c|}{\begin{tabular}[c]{@{}c@{}}Grupa\\ 3\end{tabular}}            & \multicolumn{1}{c|}{\begin{tabular}[c]{@{}c@{}}Zespół\\ 6\end{tabular}}      &  \\ \cline{1-6}
\multicolumn{1}{|c|}{\begin{tabular}[c]{@{}c@{}}Pracownia\\ FIZYCZNA\\ WFiIS AGH\end{tabular}} & \multicolumn{4}{l|}{\begin{tabular}[c]{@{}l@{}}Temat:\\ \textbf{\textit{Współczynnik załamania ciał stałych}} \end{tabular}}                                                                                                                                                                                                                                            & \multicolumn{1}{c|}{\begin{tabular}[c]{@{}c@{}}Nr ćwiczenia:\\ 51\end{tabular}} &  \\ \cline{1-6}
\multicolumn{1}{|l|}{\begin{tabular}[c]{@{}c@{}}Data wykonania:\\ 14.11.2016\end{tabular}}      & \multicolumn{1}{c|}{\begin{tabular}[c]{@{}c@{}}Data oddania:\\ 16.11.2016\end{tabular}} & \multicolumn{1}{l|}{\begin{tabular}[c]{@{}l@{}}Zwrot do poprawki:\\ \phantom{data poprawki}\end{tabular}} & \multicolumn{1}{l|}{\begin{tabular}[c]{@{}l@{}}Data oddania:\\  \phantom{data oddania}\end{tabular}} & \multicolumn{1}{l|}{\begin{tabular}[c]{@{}l@{}}Data zaliczenia:\\  \phantom{data zaliczenia}\end{tabular}} & \multicolumn{1}{l|}{\begin{tabular}[c]{@{}l@{}}OCENA:\\ \phantom{ocena}\end{tabular}}       &  \\ \cline{1-6}
                                                                                               &                                                                                         &                                                                                     &                                                                                &                                                                                   &                                                                               & 
\end{tabular}
\end{table}

\begin{center}
\begin{LARGE}
\textbf{Ćwiczenie nr 51: Współczynnik załamania ciał stałych}
\end{LARGE}
\end{center}

\section{Cel ćwiczenia}


\section{Układ pomiarowy}


\section{Wykonanie ćwiczenia}
 
 
\section{Opracowanie wyników pomiarów}

\section{Wnioski}
\begin{itemize}
\item 
\item 
\item  
\end{itemize}
\end{document}